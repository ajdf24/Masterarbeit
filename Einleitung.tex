\section{Einleitung}
In den letzten zwanzig Jahren, hat sich die Bedienung von Computern grundlegend ge�ndert. Vor nicht all zu vielen Jahren, gab es nur die M�glichkeit mit Hilfe von Maus und Tastatur mit einem Computer zu interagieren. 

Mit dem Aufkommen von Touch-Screens jedoch, �nderte sich auch die Benutzung von Computern. Es war nun m�glich direkt mit dem Bildschirm zu interagieren ohne den Umweg �ber die Maus.

Als dann wenig sp�ter die ersten Sprachsteuerungen auf den Markt kamen, �nderte sich die Interaktion mit dem Computer erneut. So ist es nun m�glich Computern mittels Sprache Befehle zu erteilen oder Texte zu sprechen, welche automatisch transkribiert werden.

Heute ist es mit manchen Smartphones schon m�glich Nachrichten wie SMS zu schreiben und zu versenden, ohne das Telefon �berhaupt in die Hand zu nehmen. Dies ist nur durch neuste Entwicklungen in der \ac{KI} m�glich.

Im selben Zeitraum hat sich parallel auch die \ac{VR} entwickelt. Sie versucht Menschen mit Hilfe von verschiedenen Brillen in eine virtuelle Welt zu versetzen. Die Entwicklungen solcher \ac{VR} Systeme wurde vor allem von der Spieleindustrie getrieben, da sie immer neue Versuche unternimmt Spieler besser in die Welt des Spiels zu versetzen. 

Eine Abstufung der \ac{VR} ist die \ac{AR}, welche versucht die analoge, reale Welt um digitale Inhalte zu erweitern. Hierbei sind die M�glichkeiten f�r den Einsatz von \ac{AR} fast unbegrenzt. 
Es ist also quasi m�glich, jede menschliche T�tigkeit durch die \ac{AR} zu unterst�tzen. 

Genau hier setzt der Schwerpunkt der Arbeit an. Im Verlauf soll versucht werden ein \ac{NUI} unter Zuhilfenahme moderner \ac{AR} und \ac{KI} Technologien zu erstellen.

Hierf�r werden aktuelle \ac{AR} und \ac{KI} Systeme daraufhin untersucht, wie sie im Zusammenspiel ein \ac{NUI} bilden k�nnen, welches allein durch Sprache und Gesten mit einem Computer interagiert.

Es soll ein �bliche T�tigkeit mit Hilfe einer \ac{AR} Technologie in der realen Welt abgebildet werden, welche dann von einem Menschen nur unter Zuhilfenahme von Sprache und Gesten gesteuert und bearbeitet werden kann.