\section{Zusammenfassung}
Die vorliegende Arbeit besch�ftigte sich mit der Frage, ob es m�glich ist, mit aktuellen \ac{AR} und \ac{AI} Technologien ein \ac{NUI} vor erstellen, welches allein durch Gesten und Sprachbefehle gesteuert werden kann.

Hierbei wurde zuerst die Frage gekl�rt, was genau ein \ac{NUI} ist. In diesem Zusammenhang wurde gezeigt, dass auch ein \ac{NUI} anhand der EN ISO 9241 validiert werden kann.

Beginnend mit einem �berblick zu Mixed Reality worden sp�ter die beiden Ger�tetypen Handheld und Video-See-Through-Display genauer beschrieben. Es wurde untersucht, auf welchem Ger�t sich ein \ac{NUI} mit Gesten und Sprache am besten umgesetzt werden kann. Die Wahl fiel auf die HoloLens, welche aktuell die beste Hardware- und Software-Unterst�tzung besitzt.

Als n�chstes wurden unterschiedliche \ac{AI}-Technologien verglichen. Der Fokus lag hierbei auf den vier aktuellen Digitalen Assistenten Amazon Alexa, Apple Siri, Microsoft Cortana und dem Google Assistent. Es wurde hier auf die Funktionsweise und eine m�gliche Anbindung �ber APIs eingegangen, wobei sich zeigte das Siri und Cortana aktuell durch nicht �ffentliche Schnittstellen nicht in Frage kommen. Der Google Assistent und Alexa sind aktuell technologisch gleich auf und k�nnen �hnlich viel leisten.

Nachdem die technischen Grundlagen und M�glichkeiten evaluiert waren, wurde ein konkretes \ac{NUI} beschrieben. Aus verschiedenen beschriebenen m�glichen Einsatzgebieten wurde ein Architektureditor gew�hlt, welcher prototypisch stark vereinfacht wurde, um zu zeigen, dass es machbar ist ein \ac{NUI} auf Basis der HoloLens zu entwickeln. 

Es wurden m�gliche Gesten und Sprachbefehle beschrieben, die es m�glich machen, einen nach dem Minecraft-Prinzip funktionierenden Block-Editor zu bedienen. 

In der Entwicklungsphase wurde zuerst auf die grundlegende Entwicklung einer Unity-Szene f�r die HoloLens eingegangen. Daraufhin wurde festgestellt, dass die HoloLens die f�r das \ac{NUI} entwickelten Gesten nicht umsetzen kann. Aus diesem Grund musste die Gesten-Bedienung f�r die HoloLens umgearbeitet werden. Anhand von Code-Beispielen wurde gezeigt, wie die Implementierung genau funktioniert. Abschlie�end wurde auf Probleme bei der Entwicklung und m�gliche L�sungen eingegangen.

Beim abschlie�enden Fazit wurde darauf eingegangen, welche grundlegenden Probleme sich w�hrend der Erarbeitung des \ac{NUI} auftraten und wie sie sich l�sen lie�en. 